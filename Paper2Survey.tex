%New Section: Securing Internet of Things with Software Defined Networking
\section {Securing Internet of Things with Software Defined Networking}
%New Subsection: Overview
\subsection {Overview}
\smallskip

The Internet of Things (IoT) is becoming more relevant to networking technology considerations as time goes on. The reason for this is due to the number of devices that are being developed that allow users to connect to them from just about anywhere. On top of that, these devices are generally designed with minimal security considerations and can be vulnerable. Due to their inherent nature, networks must be able to accommodate devices that not only generate big data and communication overhead but are also susceptible to many vulnerabilities. 

\smallskip

The authors surveyed multiple Software Defined Networking architectures that were designed to extend security mechanisms into the IoT environment. They review each one individually and point out their benefits and their drawbacks. Based on the information they gathered, the authors then propose their own SDN architecture for the IoT environment that helps provide a more well-rounded solution with more of the benefits and less of the drawbacks.

\smallskip

"Since IoT has scalability and heterogeneity challenges, researchers and service providers have been exploring alternative solutions, such as SDN, to increase IoT's bandwidth and flexibility." \cite {kalkan2017securing} The reason IoT poses such a challenge for regular architectures is due to the diversity of devices and the fact that they are rather resource-constrained. IoT is also increasing at an alarming rate which in turn causes issues due to the amount of communication and the data that must be processed and stored. As more devices continue to connect, current architecture may not be able to handle the requirements that will be enlisted.

\smallskip

The main drive behind the paper was to leverage SDN to separate the control and data planes. This will free up many network devices because the SDN controller will oversee all decision making while switches will oversee just forwarding the data. Another issue was that in standard networks, routers are generally configured with individual mechanisms to handle security, link failure and many of the different forwarding mechanisms. When any of these need to be updated, management is done on an individual basis but by leveraging SDN, one would be able to establish a centrally managed system to combat this issue. 

\smallskip

Essentially, SDN is being leveraged to “provide a network environment that is more dynamic and agile.” \cite {kalkan2017securing} This will address the initial issues of heterogeneity and scalability of IoT devices. By separating the data and control planes and allowing the usage of less complex network devices as well as allowing for central management, SDN is able to provide many benefits and a feature rich toolset that can prove to be beneficial to the standard network. To avoid bottlenecking, SDN is able to provide logical routing and avoid this issue by maintaining a view of the entire network. . “SDN integration also simplifies information analysis and decision-making processes in IoT.” \cite {kalkan2017securing} On another note, the authors note how SDN can be used in the IoT environment to help with troubleshooting and debugging due to the tools it naturally features.

%New Subsection: Identity-Based Authentication Scheme for IoT Based on SDN
\subsection {Identity-Based Authentication Scheme for IoT Based on SDN}
%New Subsubsection: Overview
\subsubsection {Overview:}
\smallskip

The first paper they surveyed proposed an “identity-based authentication scheme for IoT based on SDN.” \cite {kalkan2017securing} The original authors aimed at leveraging the SDN controllers overview of the network to help determine any misbehavior on the network all while reducing the overhead generated by security-related information that is distributed across the network. The idea of implementing SDN into an IoT network environment is centralize control and be able to manage some of the complexities that will surface in this IoT environment. For this survey, the original authors decided to use IPv6 in order to have a suitable identity scheme. The SDN controller will be used to assign virtual IPv6-based identity to IoT devices. \cite {salman2016identity} For key establishment, Elliptic-curve cryptography (ECC) will be utilized as it rather efficient and is the most commonly used method for IoT. 

%New Subsubsection: How it Works
\subsubsection {How it Works:}
\smallskip

For this survey, the authors assumed “that the root controller public key is hardcoded in each thing when manufactured. “\cite {salman2016identity} The controller will create public keys for the devices that are using ECC while gateway will have its own set of public and private keys using ECC. 

\smallskip

The first phase will require that the gateway send its ID and corresponding public key to the controller and will in turn receive a public key certificate signed by the controller’s private key in order to authenticate itself to the controller \cite {salman2016identity}. The second phase is where the devices will register with the gateway by sending an authentication request. This will include both that devices unique ID along with a random nonce. The nonce will then be used as the private key and be encrypted by the controller’s public key. To prevent replay attacks, a second nonce will be sent. A preliminary identity check will be done by the gateway to verify whether the technology-specific address match one in the registration request. \cite {salman2016identity} If the initial verification is successful, the gateway then forwards the request, along with the corresponding certificate, to the controller where an IPv6 and a public key using ECC will be generated along with a private key that will be based on the random nonce that was received. The controller will then forward this to the gateway and from there, the gateway will “decrypt and store the hash identify and the corresponding public key.” \cite {salman2016identity} The gateway then sends the “devices public key, the hashed address, and the gateway’s public key encrypted by the devices private key.” \cite {salman2016identity} The device will now verify registration with the controller and the authentication phase begins.

\smallskip

The third phase will be where the device will authenticate itself. It does this by sending a random none along with the hashed IPv6 address to the gateway. This will all be signed by the gateway’s public key to ensure authentication. Once the gateway receives and authenticates, it sends back the original nonce sent from the device along with a second nonce and both will be encrypted separately using the device’s public key. For the device to authenticate, it will return the second nonce, the one generated by the gateway during response, and encrypt It using the gateway’s public key. The device is now authenticated and has permission to access the network. Since the device is authenticated, gateways can refer to the controller if the device were to move domains and need to connect to another gateway. 
This proposal was never deployed nor tested to try and see how well it operates in such an environment. 

%New Subsection: Host-Based Intrusion Detection and Mitigation Framework for IoT Based on SDN
\subsection {Host-Based Intrusion Detection and Mitigation Framework for IoT Based on SDN}
%New Subsubsection: Overview
\subsubsection {Overview:}
\smallskip

The second paper surveyed is proposed as a “host-based intrusion detection and mitigation framework for IoT based on SDN”, also known as IoT-IDM. \cite {kalkan2017securing} The main drive behind the survey was to devise a more secure framework that would allow for scalability while remaining affordable. Their proposal would allow for detection and mitigation at a network-level, rather than at the endpoints. One of their requirements during design was to devise a proposal that would detect attacks and then automatically trigger the appropriate countermeasures once the attack has been identified. While this is a fantastic idea, the authors had to ensure that efficiency was next on their list of design requirements to ensure overhead would not become an issue. Finally, they required that their proposal be scalable and be able to support new devices and their new technologies for both today and in the future.

%New Subsubsection: How it Works
\subsubsection {How it Works:}
\smallskip

Their framework focuses on the use of two main technologies: Intrusion Detection System (IDS) and OpenFlow. The IDS can be either network-based or it can be host-based. In a network-based IDS, the traffic going across a network segment will be monitored and it will analyze traffic on the network and look for anything unusual or suspicious. On the other hand, a host-based IDS will only monitor a single-host in a network. This includes reviewing activities that the host is performing and remaining aware of its characteristics. The other technology being used is OpenFlow. This is being used “to enable the communication interface between the controller and underlying switches.” \cite {nobakht2016host}

\smallskip

The IoT-IDM architecture is broken down into five main modules: Device Manager, Sensor Element, Feature Extractor, Detection Unit, and Mitigation. \cite {nobakht2016host} 
The Device Manager will maintain a database of all known devices and will include a risk assessment for each device. The assessment will note “potential security risks associated with them, the detection method of known attacks, and if it is applicable, appropriate defense mechanisms.” \cite {nobakht2016host} The Sensor Element will be used to monitor a targeted device’s activities while connected to the network by building a virtual inline sensor on top of the SDN controller. OpenFlow will be utilized to help create and distribute rules to try and redirect traffic from the device and network toward the sensor element. This will lower the amount of processing and load on the controller since only traffic for a targeted device will be passing through the sensor element and only the data collected from that targeted device is logged to be used later for auditing of incidents. The Feature Extractor used the data captured by the sensor element to determine the features of a specific device. The feature extractor is highly dynamic and allows for the extraction of specified data depending on the circumstances. Data captured and extracted by the feature extractor will be used later on by the detection unit. This system revolves around extensive knowledge of individual hosts on the network along with known possible attacks, but the problem lies with those attacks that may be unknown and how the system will handle them. The Detection Unit examines and identifies the data captured by the sensor elements and the information retrieved by the feature extractor to try and determine if there is any suspicious activity on the network. SDN is leveraged here to “enable machine learning algorithms to build predictive models for detecting malicious traffic” in the IoT-IDM architecture. \cite {nobakht2016host} If the detection unit determines that suspicious activity has been detected on the network, it identifies the target and the source of the activity. From there, it will either send out an alert stating that suspicious activity was found but mitigation is not possible, or it will reveal any information regarding the activity to the mitigation module. The Mitigation Module is responsible for deploying countermeasures once a harmful attack has been identified. OpenFlow is leveraged to block or redirect the traffic flow and also allows a compromised host to be quarantined and provide limited access to the device. 

%New Subsubsection: Drawbacks
\subsubsection {Drawbacks:}
\smallskip

A main concern for this design is that since it is positioned on top of the SDN controller, “it is not technically feasible to use IoT-IDM for intrusion detection and mitigation process of all devices in a home network.” \cite {nobakht2016host} There would be too much overhead that can be added if too many devices were utilizing this architecture and would cause a rather negative experience if it were to be utilized on a larger scale. Also, since this is a host-based IDS, it means that the network is still susceptible to malicious traffic and protection is only guaranteed for a targeted host. 

%New Subsection: Black SDN for the Internet of Things
\subsection {Black SDN for the Internet of Things}
%New Subsubsection: Overview
\subsubsection {Overview:}
\smallskip

The third paper they surveyed proposed Black SDN for IoT, an SDN-based secure networking mechanism. The original authors chose to obscure all data that is transmitted across the network by encrypting both the payload and the header. Although this poses some challenges regarding routing the packets, the original authors utilized an SDN controller to control a broadcast routing protocol. This mechanism helps mitigate against analysis of the network’s traffic but also against attacks that are focused on gathering data.

%New Subsubsection: How it Works
\subsubsection {How it Works:}
\smallskip

Once a packet is completely encrypted, including both the payload and the header, it is considered a black packet. Black packets can only be routed by the SDN controller and the routes can be determined either statically or dynamically. In this design, the authors leverage the sleep and awake cycles of battery-constrained nodes. If the route is statically set, the route is predetermined and cannot be altered after transmission. If the route is dynamically set, a packet that comes across a node caught in sleep mode will be redirected to a node that is awake.
%New Subsubsection: Benefits and Drawbacks
\subsubsection {Benefits and Drawbacks:}
\smallskip

Although the idea of obscuring entire packets sounds appealing, the original authors did a poor job in designing the proper mechanism for obscuring the packets. Since the packets lack unique identifiers, only SDN controllers can route black packets and this may be a cause for concern later down the road. Another issue is that since the controller is transmitting and routing against nodes depending on their sleep cycle, communication overhead is incurred. Overall, the concept was there but the original authors failed to present an ideal mechanism for obstruction of entire packets including header and payload. 

%New Subsection: Flow-Based Security for IoT Devices Using an SDN Gateway
\subsection {Flow-Based Security for IoT Devices Using an SDN Gateway}
%New Subsubsection: Overview
\subsubsection {Overview:}
\smallskip

The fourth paper they surveyed proposed “a flow-based security approach for IoT devices using an SDN gateway.” \cite {kalkan2017securing} This is done by configuring the SDN controller as a gateway and then monitoring all traffic that flows across the controller. If any suspicious activity is found, a response for the corresponding flow will be triggered. The main drive behind this proposal was the “need for a flexible and dynamic method of IoT security.” \cite {bull2016flow}

%New Subsubsection: How it Works
\subsubsection {How it Works:}
\smallskip

The SDN controller acting as a gateway is made up of three main components: switch functionality, a statistics manager, and mitigation actions. The statistics manager is in charge of data collection and being able to characterize flows in order to help detect anomalous flows. When an anomaly has been detected, an action is selected from the mitigation action list and is enforced. “The three possible actions are; Block, Forward, or Apply QoS.” \cite {bull2016flow} Block will prevent a device from communicating across the network by blacklisting it. Forward redirects the traffic to a quarantined section that is configured on the network until further review can be done to determine the appropriate actions to take with the device that triggered the mitigation action. Apply QoS is used when the controller cannot determine what the best choice is or if it is unable to block a single source. This will limit general service requirements such as bandwidth in order to limit the flow.

%New Subsubsection: Benefits and Drawbacks
\subsubsection {Benefits and Drawbacks:}
\smallskip

While this proposal may have lacked valuable information to validate its effectiveness, the original authors were able to propose an additional option to use SDN in order to increase security. The original tests that were done using this design were only using Transmission Control Protocol (TCP) and Internet Control Message Protocol (ICMP) DDoS attacks. Because of this, the design does not have sufficient testing done for validation of its effectiveness.

%New Subsection: SDN-Based Architecture for IoT and Improvement of the Security
\subsection {SDN-Based Architecture for IoT and Improvement of the Security}
%New Subsubsection: Overview
\subsubsection {Overview:}
\smallskip

The fifth paper the authors surveyed proposes a secure SDN-based solution by utilizing multiple controllers for a single OpenFlow enabled switch. The main drive behind this design was that current mechanisms that are deployed on the Internet edge are not sufficient to handle the future of the Internet. “The borderless architecture of the IoT raises additional concerns over network access control and software verification.” \cite {flauzac2015sdn} This design utilizes multiple SDN controllers to be synced together and setup “in a security perimeter enabling a granular control over network access and continuous monitoring of network endpoints.” \cite {flauzac2015sdn}

%New Subsubsection: How it Works
\subsubsection {How it Works:}
\smallskip

The original authors of this design set forth guidelines to ensure their design would be successful. The first issue they addressed was the single-point-of-failure issue and this was done by utilizing multiple controllers where there will always be a controller ready to take over in the event that a controller was to fail. OpenFlow enabled switches are then configured to more than one controller to satisfy the single-point-of-failure concern. Since there are multiple controllers, OpenFlow can either be in equal interaction mode or master/slave interaction mode. With equal interactions, read and write access on the switch is given to all the controllers which will require them to sync up in order for them not to overlap during read or writes. With master/slave interaction, one controller will be labeled as the master and all others will be labeled as slaves. This prevents controllers trying to read or write over each other. 

\smallskip

For their design, they classify nodes into two categories: nodes can be considered an OpenFlow node if they have the resources available and if they don’t, they are considered a smart object. Regarding SDN domains, each one will have a dedicated SDN controller that will handle all traffic within its domain. The authors then propose a new type of controller, known as the root controller or border controller, to achieve large scale interconnection across multiple SDN domains. \cite {flauzac2015sdn} This controller will be responsible for communication and connection across other SDN border controllers. Another item to note is that the authors chose not to distribute control functions and instead decided to distribute the routing functions and security rules of each border controller. \cite {flauzac2015sdn}

\smallskip

During the initial setup, the OpenFlow switch will be authenticated by the SDN controller. Once authenticated, “the controller blocks switch ports directly connected to the users.” \cite {flauzac2015sdn} Then, the controller only attempts to authenticate the users by only allowing the users authenticate traffic to flow across the network. Each user is given a level of authorization so once the user passes authentication, the controller will look at what their level of authentication is and then push the corresponding flow entries to the access switch. \cite {flauzac2015sdn} For IoT devices, each must be associated with an OpenFlow enable node that is connected to at least one controller within the domain.

%New Subsubsection: Benefits and Drawbacks
\subsubsection {Benefits and Drawbacks:}
\smallskip

While this design allows for each domain to remain independent in the event of a failure, its overall structure will cause communication delays. The design also revolves around the Grid of Security concept which essentially allows SDN domains to remain independent and enforce whatever security rules they choose but when a node wants to communicate with another node in separate domain, the flow must be directed to the border controllers, otherwise considered the security controllers, and then that controller is in charge of communication, only allowing recognized traffic and a safe connection. Unfortunately, the original authors did not do any performance testing which gives no information as to how much communication overhead is truly being generated with this design.

%New Subsection: Proposed Architecture (Rol-Sec)
\subsection {Proposed Architecture (Rol-Sec)} 
%New Subsubsection: Overview
\subsubsection {Overview:}
\smallskip

After surveying the proposals, the authors concluded that these mechanisms could be classified into three different categories: network-based, traffic-based, and crypto-based. The network-based mechanisms were designed to “deal with the architectural structure of the network elements.” \cite {kalkan2017securing} Traffic-based mechanisms were mainly focused on flows and were there for detection and prevention of malicious activity on the network by analyzing flow on the network. Crypto-based solutions provided security by leveraging encryption as the focus of their proposal. By utilizing the advantages and drawbacks of each mechanism, as well as utilizing the classifications as a guide, the authors propose their own take on a mechanism that leverages SDN to secure IoT devices and environments. 

\smallskip

The authors proposed Rol-Sec, a role-based security controller for the SDN-IoT environment. By assessing the advantages and disadvantages of each proposal they reviewed, the authors were able to devise a mechanism with minimal drawbacks. Rol-Sec consists of utilizing an SDN-IoT gateway (SDN-IoT GW) in each domain that will not only communicate within its own domain but will also communicate across multiple domains via other SDN-IoT GW’s while being managed by several controllers. 

\smallskip

For their architecture, the authors decided on using role separations of the controllers in order to try and distribute overhead generated by IoT data. The three controllers will be an intrusion controller, a key controller, and a crypto controller. Each one has an individual task and essentially distributes the overhead across multiple controllers.

\smallskip

The intrusion controller is responsible for availability and secure routing. Availability is accomplished by analyzing the flow and the corresponding rules for each flow, along with monitoring the general traffic. If some sort of intrusion were to be sent into the network, the intrusion controller would be able to detect it and then attempt to mitigate it. As for secure routing, the intrusion controller will attempt to find not only the best possible route but also a route that is secure. This is done by communicating with the key controller and receiving whatever keys may be needed for the utilized algorithm. 

\smallskip

The key controller will act as the key manager will both store and distribute keys. On top of acting as repository and a trusted third party, the key controller is also responsible for cryptographic operations that may be needed by either the intrusion controller of the crypto controller. This controller can communicate with both the intrusion controller and the crypto controller.

\smallskip

The crypto controller is responsible for ensuring “integrity, confidentiality, privacy, authentication, and identity management.” \cite {kalkan2017securing} Different cryptographic algorithms are leveraged to “take into consideration the dynamic, heterogeneous, and scalable characters of the IoT environment via the SDN controllers.” \cite {kalkan2017securing} Since the crypto controller require keys during its operation, it must be able to communicate with the key controller.

%New Subsubsection: Benefits and Drawbacks
\subsubsection {Benefits and Drawbacks:}
\smallskip
By designing a distributed architecture that is devised of multiple controllers with specified roles, the authors were able to mitigate the single-point-of-failure issue. For this design, the authors also stated that since the keys will be "stored and distributed over different communication links, they do not need memory storage on other controllers and they do not affect cryptographic operations or intrusion detection or prevention mechanisms." \cite {kalkan2017securing} Overall bandwidth is improved as well since the communication traffic will be distributed and not originating from a single point or communication link.

\smallskip

Since their design is accomplishing by role separation and the distribution of multiple controllers, communication channels across the devices will need to be used instead of single localized system that has the information in memory. Because of this communication requirements, their design incurs additional latency and communication overhead. To mitigate this, the authors suggest utilizing powerful devices that can process at high rates along with utilizing high speed link connections. 

%New Subsection: Observations
\subsection {Observations}
\smallskip

Although each proposal has its own advantages and disadvantages, there are still concerns for the future of SDN-IoT security. One of the concerns is that SDN is susceptible to the single-point-of-failure issue and it becomes an issue since SDN is centrally managed. This means that if a controller was to have an availability issue, every single system in the architecture will become unavailable as well. Another concern is that the limited table sizes of the switches will become an issue as the number of devices in the environment grows. This is because unknown flows are individually inserted as a new entry into the memory of that switch and as the number of devices added to the environment increases, efficiently trying to store the drop and forward rules becomes increasingly difficult. Further concern surrounds the possibility of causing a bottleneck in the system between the gateway and the controller as the number of IoT devices increases. Although there are challenges for the SDN-IoT environment, it is fair to say that SDN can be leveraged to aid security requirements in such an environment as long as consideration of its inherent vulnerabilities are noted, and the appropriate measures are taken to reduce their risk.
