%New Section: Improving Wireless Privacy with an Identifier-Free Link-Layer Protocol
\section {Improving Wireless Privacy with an Identifier-Free Link-Layer Protocol}
%New Subsection: Overview
\subsection {Overview}
\smallskip

When talking about the link-layer, security is generally not one of the terms to come up in conversation. The reason for this is because the link-layer does not really address security concerns the way it could. As technology continues to move forward, it is crucial to ensure that everything that makes technology a success also moves forward, in parallel. This does not mean having to devise a completely new protocol, but it must at least be taken into consideration. The following paper was a fantastic approach to addressing a security issue that is present in link-layer communications. The authors propose an Identifier-Free Link-Layer Protocol, known as SlyFi, which aims to increase privacy by obscuring bits that are transmitted.

\smallskip

The main motivation behind the new protocol was to increase security across wireless technologies. The wireless ecosystem continues to grow at an alarming rate and it is becoming difficult to ensure that all these devices remain secure. Many security products are for endpoint devices, not dealing with the transmission of identifiers across foreign nodes. Since wireless device generally broadcast their messages and anyone listening can try to intercept these packages, there is a major concern for connection establishments and their authenticity. This lack of security across identifiers that are used to establish connections on wireless devices was the major motivation for devising a new link-layer protocol. 

\smallskip

Although there are protocols, such as WPA, that attempt to address these concerns, they are only able to encrypt the contents of the message, not the identifiers in the header. With low-level identifiers still visible, an attacker listening to packets would be able to inventory, track, and profile users and devices without them even knowing. These attackers do not even have to be present to exploit this vulnerability making it a much bigger cause for concern. With these identifiers still visible, privacy concerns become an issue and, in a world, where people have many more than one device, the stakes continue to rise.

\smallskip

Their goal was to try and limit the linkability of communication packets. The authors first wanted to ensure that no information should be noted on the packet that would be able to tell someone who the sender or the intended receiver is. The other was avoid an attacker from profiling, fingerprinting, and tracking packets from the same sender. By doing so, they would able to completely obscure link-layer communication and increase privacy over wireless networks.

%New Subsection: Security Requirements
\subsection {Security Requirements}
\smallskip

To design a protocol that will be superior to those currently in rotation, the authors listed security requirements that were to be used as a guide. 
First on their list of security requirements is strong unlinkability. The motivation behind this was to ensure packets could not be linked together for an attacker to track or profile them. In other words, this means is that only the parties communicating should be able to determine who the sender or receiver of a packet is. To accomplish this, the identifiers in the headers cannot be consistent and must be changed and randomized for every packet.

\smallskip

The next security requirement is in regard to authenticity. According to the authors, “to restrict the discovery of services to authorized clients and prevent spoofing and man-in-the-middle attacks, recipients should be able to verify a message’s source. More formally, B should be able to verify that A was the author of c and that it was constructed recently (to prevent replay attacks).” \cite {greenstein2008improving}

\smallskip

Following this, we find confidentiality as the next requirement on their list. Essentially, the authors state that only the parties communicating should be able to see neither the payload of a packet nor the header and its fields. Last on the list we have message integrity. The authors want to ensure that anyone who receives a message should still be able to determine if it was tampered with in any way since it was sent from its original destination. 

%New Subsection: How it Works
\subsection {How it Works}
\smallskip

By using their security requirements as a guide, the authors were able to devise a new protocol, known as SlyFi, that would satisfy all their requirements and even go beyond that. SlyFi is able to satisfy the security requirements by leveraging two mechanisms for concealing identifiers called Tryst and Shroud. SlyFI was meant to replace 802.11 by encrypting entire packets and then obscuring the header using pseudorandom identifiers. “A client wishing to join and send data to a SlyFi network sends a progression of messages similar to 802.11. Instead of sending these messages in the clear, they are encapsulated by the two identity-hiding mechanisms.” \cite {greenstein2008improving} 

\smallskip

Using SlyFi, if someone was looking to connect to a wireless access point, their device would transmit discovery messages, known as probes, that would be encapsulated using Tryst. These messages would be “encrypted such that: 1) only the client and the networks named in the probe can learn the probe’s source, destination, and contents, and 2) messages encapsulated for a particular SlyFi AP sent at different times cannot be linked by their contents.” \cite {greenstein2008improving} Once an access point receives one of these messages, it verifies the user that generated the message and makes sure they are authorized and also sends back an encrypted reply that will tell that users device that the access point is active. After this, if the user wants to establish a connection then an authentication request would be sent. This request would also be encapsulated using Tryst, but it would also contain information to help initialize the connection such as “keys for subsequent data transmission, which are used to bootstrap Shroud.” \cite {greenstein2008improving} After a communication channel is established, Shroud will obscure both the payload and the header for everything sent. This will ensure that nobody will be able to see the destination or source addresses and be able to create a link to multiple packets as well as not being able to see the contents of the packets. Tryst and Shroud will essentially encrypt messages and their corresponding headers. Although Tryst and Shroud are rather similar in their functions, “the essential differences between them arise due to the different requirements of discovery, link establishment, and data transfer.” \cite {greenstein2008improving}

\smallskip

According to the authors, “Identifiers are used in wireless protocols for two general functions: 1) as a handle by which to discover a service and establish a link to it, and 2) to address packets on a link and allow unintended recipients to ignore packets efficiently.” \cite {greenstein2008improving} Both Tryst and Shroud are able to satisfy both of the requirements needed by unobscured identifiers. 

\smallskip

When discovery or binding messages are used, Tryst will be the mechanism in charge of ensuring that only the devices communicating, and trusted network devices are able to decipher the header information as well as the contents of the message. Tryst is built upon symmetric key encryption and benefits from the infrequent communication and narrow interface features of symmetric key encryption. Since devices are not usually sending these discovery and binding messages, they are considered infrequent. It is only when a device is searching for an access point that the messages are sent. Once established, they no longer need to be sent. Since the binding process is so quick, the narrow interface and short time of life allows for these messages to avoid being able to be linked together.

\smallskip

Once the connection has been established, it is time for Shroud to take over. The reason that a second mechanism is needed is due to the features that Tryst leverages in its mechanism which does not work for packages containing actual data. Shroud leverages public key encryption mechanisms to provide strong unlinkability for data transport packets. Shroud ensures that an address will never appear in two separate messages and allow them to be tracked by generating a pseudo-random identifier which is then used as an initialization vector for payload encryption. 

%New Subsection: Performance
\subsection {Performance} 
\smallskip

The authors decided to use multiple protocols as benchmarks to compare against SlyFi. The protocols that were tested were wifi-open, wifi-open-driver, wifi-wpa, public key, symmetric key, and armknecht. Their first major test was to determine discovery and link setup performance by testing how long it took to establish a connection before a client was able to start sending data. Their results proved to be positive in comparison to public key and symmetric key, Tryst had “faster link setup times than wifi-wpa and scales as gracefully as wifi-open when varying each of these parameters.” \cite {greenstein2008improving} To take things further, they managed to load multiple keys onto an access point which mimics an active access point. The reason for this is because an access point in a real-world environment would manage many keys which in turn would impact the link setup time. The results of this showed that Tryst was still faster that public key and symmetric key. As the number client keys on the access point went up, so did the time it took symmetric key to establish a link. Another portion of the test included the number of probes to unknown networks before a connection is established. The results to this were positive, showing no substantial increase in the time to setup as the number of probes increased unlike symmetric key and public key that both grew rather steadily as the number of probes went up.

\smallskip

The second major test was to determine how well Shroud performs in terms of data transport. Shroud was tested on both a software and a hardware platform to get a better comparison. The round-trip times of ICMP messages showed high latency for Shroud based on the software platform but on the hardware platform, it performed just as well as all other in the lineup. They believe that the reason Shroud based on the software platform was so high was because of an inefficiency in the Click runtime. Unfortunately, there was no further testing done to determine if this was true.

%New Subsection: Observations
\subsection {Observations}
\smallskip

Overall, SlyFi is a well-documented and proposed identifier-free link-layer protocol that has great potential to increase privacy for wireless devices. It helps increase privacy by removing traceable identifiers from link-layer packets and using a unique proprietary mechanism to encrypt both the header and the payload. As the number of wireless devices in our society continues to grow, a protocol such as SlyFi becomes more and more important. Overall, I believe they did a fantastic job and the results speak for themselves.

